\documentclass[10pt]{article}
\usepackage[utf8]{inputenc}
\usepackage[T1]{fontenc}

\begin{document}
[21] S. C. Kleene, Recursive functionals and quantifiers of finite types I, Trans. Amer. Math. Soc. 91 (1959) 1-52.\\[0pt]
[22] S. C. Kleene, Recursive functionals and quantifiers of finite types II, Trans. Amer. Math. Soc. 108 (1963) 106-142.\\[0pt]
[23] K. Ko and H. Friedman, Computational complexity of real functions, Theor. Comput. Sci. 20 (1982) 323352.\\[0pt]
[24] K. Mehlhorn, Polynomial and abstract subrecursive classes, J. Comput. System Sci. 12 (1976) 147-178.\\[0pt]
[25] M. A. Nielsen and I. L. Chuang, Quantum Computation and Quantum Information, Cambridge University Press, 2000.\\[0pt]
[26] H. Nishimura and M. Ozawa, Computational complexity of uniform quantum circuit families and quantum Turing machines. Theor. Comput. Sci. 276 (2002) 147-181.\\[0pt]
[27] M. Ozawa and H. Nishimura, Local transition functions of quantum Turing machines, RAIRO - Theoretical Informatics and Applications 276 (2000) 379-402.\\[0pt]
[28] G. Peano, Sul concetto di numero, Rivista di Matematica 1 (1891) 87-102, 256-267.\\[0pt]
[29] R. I. Soare, Computability and recursion. The Bulletin of Symbolic Logic, vol. 2, No. 3 (1996), pp.284-321.\\[0pt]
[30] P. W. Shor, Polynomial-time algorithms for prime factorization and discrete logarithms on a quantum computer, SIAM J. Comput. 26 (1997) 1484-1509.\\[0pt]
[31] M. Townsend, Complexity for type-2 relations, Notre Dame J. Formal Logic 31 (1990) 241-262.\\[0pt]
[32] A. M. Turing, On computable numbers with an application to the Entscheidungsproblem. Proc. London Math. Soc. ver. 2, 42 (1936) 230-265. Errutum ibid 43 (1937) 544-546.\\[0pt]
[33] M. Villagra and T. Yamakami. Quantum state complexity of formal languages, Proceedings of the 17th International Workshop on Descriptional Complexity of Formal Systems (DFCS 2015), Springer, Lecture Notes in Computer Science, vol. 9118, pp. 280-291, 2015.\\[0pt]
[34] T. Yamakami. Structural properties for feasiblely computable classes of type two. Mathematical Systems Theory 25 (1992) 177-201.\\[0pt]
[35] T. Yamakami, Feasible computability and resource bounded topology, Inf. Comput. 116 (1995) 214-230.\\[0pt]
[36] T. Yamakami, A foundation of programming a multi-tape quantum Turing machine, in the Proc. 24th Mathematical Foundations of Computer Science (MFCS'99), Lecture Notes in Computer Science, vol. 1672, pp. 430-441, 1999. See also arXiv:quant-ph/9906084.\\[0pt]
[37] T. Yamakami. Quantum NP and a quantum hierarchy. In the Proc. of the 2nd IFIP International Conference on Theoretical Computer Science (TCS 2002), Kluwer Academic Press (under the name Foundations of Information Technology in the Era of Network and Mobile Computing), the series IFIPThe International Federation for Information Processing, vol. 96 (Track 1), pp. 323-336, 2002.\\[0pt]
[38] T. Yamakami, Analysis of quantum functions, Int. J. Found. Comput. Sci. 14 (2003) 815-852. A preliminary version appeared in the Proc. 19th International Conference on Foundations of Software Technology and Theoretical Computer Science, Lecture Notes in Computer Science, Vol. 1738, pp. 407-419, 1999.\\[0pt]
[39] T. Yamakami, A recursive definition of quantum polynomial time computability (extended abstract), in the Proc. of the 9th Workshop on Non-Classical Models of Automata and Applications (NCMA 2017), Österreichische Computer Gesellschaft 2017, the Austrian Computer Society, pp. 243-258, 2017.\\[0pt]
[40] A. C. Yao, Quantum circuit complexity, in the Proc. of the 34th Annual IEEE Symposium on Foundations of Computer Science (FOCS'93), pp. 80-91, 1993.


\end{document}