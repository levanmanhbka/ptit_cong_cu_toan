\documentclass[10pt]{article}
\usepackage[utf8]{inputenc}
\usepackage[T5]{fontenc}
\usepackage[vietnam]{babel}

\begin{document}
[21] S. C. Kleene, Hàm đệ quy và lượng từ của các loại hữu hạn I, Trans. Amer. Math. Soc. 91 (1959) 1-52.\\[0pt]
[22] S. C. Kleene, Hàm đệ quy và lượng từ của các loại hữu hạn II, Trans. Amer. Math. Soc. 108 (1963) 106-142.\\[0pt]
[23] K. Ko và H. Friedman, Độ phức tạp tính toán của các hàm số thực, Theor. Comput. Sci. 20 (1982) 323-352.\\[0pt]
[24] K. Mehlhorn, Các lớp con đệ quy đa thức và trừu tượng, J. Comput. System Sci. 12 (1976) 147-178.\\[0pt]
[25] M. A. Nielsen và I. L. Chuang, Tính toán Lượng tử và Thông tin Lượng tử, Cambridge University Press, 2000.\\[0pt]
[26] H. Nishimura và M. Ozawa, Độ phức tạp tính toán của các họ mạch lượng tử đồng nhất và máy Turing lượng tử. Theor. Comput. Sci. 276 (2002) 147-181.\\[0pt]
[27] M. Ozawa và H. Nishimura, Các hàm chuyển tiếp cục bộ của máy Turing lượng tử, RAIRO - Tin học Lý thuyết và Ứng dụng 276 (2000) 379-402.\\[0pt]
[28] G. Peano, Sul concetto di numero, Rivista di Matematica 1 (1891) 87-102, 256-267.\\[0pt]
[29] R. I. Soare, Tính toán và đệ quy. The Bulletin of Symbolic Logic, vol. 2, No. 3 (1996), pp.284-321.\\[0pt]
[30] P. W. Shor, Các thuật toán thời gian đa thức cho phân tích số nguyên tố và logarit rời rạc trên máy tính lượng tử, SIAM J. Comput. 26 (1997) 1484-1509.\\[0pt]
[31] M. Townsend, Độ phức tạp cho các quan hệ loại-2, Notre Dame J. Formal Logic 31 (1990) 241-262.\\[0pt]
[32] A. M. Turing, Về các số có thể tính toán với một ứng dụng cho vấn đề Entscheidungsproblem. Proc. London Math. Soc. ver. 2, 42 (1936) 230-265. Errutum ibid 43 (1937) 544-546.\\[0pt]
[33] M. Villagra và T. Yamakami. Độ phức tạp trạng thái lượng tử của các ngôn ngữ hình thức, Proceedings of the 17th International Workshop on Descriptional Complexity of Formal Systems (DFCS 2015), Springer, Lecture Notes in Computer Science, vol. 9118, pp. 280-291, 2015.\\[0pt]
[34] T. Yamakami. Các thuộc tính cấu trúc cho các lớp có thể tính toán khả thi của loại hai. Mathematical Systems Theory 25 (1992) 177-201.\\[0pt]
[35] T. Yamakami, Tính toán khả thi và tôpô bị giới hạn tài nguyên, Inf. Comput. 116 (1995) 214-230.\\[0pt]
[36] T. Yamakami, Nền tảng của lập trình máy Turing lượng tử nhiều băng, trong Proc. 24th Mathematical Foundations of Computer Science (MFCS'99), Lecture Notes in Computer Science, vol. 1672, pp. 430-441, 1999. Xem thêm tại arXiv:quant-ph/9906084.\\[0pt]
[37] T. Yamakami. Quantum NP và một hệ phân cấp lượng tử. Trong Proc. của Hội nghị Quốc tế IFIP lần thứ 2 về Lý thuyết Khoa học Máy tính (TCS 2002), Kluwer Academic Press (dưới tên Foundations of Information Technology in the Era of Network and Mobile Computing), chuỗi IFIP-The International Federation for Information Processing, vol. 96 (Track 1), pp. 323-336, 2002.\\[0pt]
[38] T. Yamakami, Phân tích các hàm lượng tử, Int. J. Found. Comput. Sci. 14 (2003) 815-852. Một phiên bản sơ bộ xuất hiện trong Proc. 19th International Conference on Foundations of Software Technology and Theoretical Computer Science, Lecture Notes in Computer Science, Vol. 1738, pp. 407-419, 1999.\\[0pt]
[39] T. Yamakami, Một định nghĩa đệ quy về tính toán thời gian đa thức lượng tử (tóm tắt mở rộng), trong Proc. của Hội thảo lần thứ 9 về Mô hình Phi Cổ điển của Máy Tự động và Ứng dụng (NCMA 2017), Österreichische Computer Gesellschaft 2017, Hiệp hội Máy tính Áo, pp. 243-258, 2017.\\[0pt]
[40] A. C. Yao, Độ phức tạp mạch lượng tử, trong Proc. của Hội nghị chuyên đề thường niên lần thứ 34 của IEEE về Nền tảng của Khoa học Máy tính (FOCS'93), pp. 80-91, 1993.

\end{document}